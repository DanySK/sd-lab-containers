% !TeX spellcheck = en_GB
\documentclass{beamer}\mode<presentation>{\usetheme{AMSBolognaFC}}
\setbeamertemplate{bibliography item}{\insertbiblabel}

\usepackage{common}

\newcommand{\labN}{2}

\title[L\labN{} -- Containers 101]{L\labN{} -- Containers 101}
%
\subtitle[SD]{Distributed Systems / Technologies}
%
\author[Ciatto \and Omicini]
{\emph{Giovanni Ciatto} \and Andrea Omicini\\
\texttt{giovanni.ciatto@unibo.it \and andrea.omicini@unibo.it}}
%
\institute[DISI, Univ. Bologna]
{Dipartimento di Informatica -- Scienza e Ingegneria (DISI)\\\textsc{Alma Mater Studiorum} -- Universit{\`a} di Bologna a Cesena}
%
\date[A.Y. 2021/2022]{Academic Year 2021/2022}

\setbeamercovered{transparent}

\AtBeginSection[]
{
\begin{frame}[c]\frametitle{Next in Line\ldots}
%    \begin{multicols}{2}
        \tableofcontents[sectionstyle=show/shaded, subsectionstyle=show/hide, subsubsectionstyle=hide/hide]
%    \end{multicols}
\end{frame}
}

\begin{document}

\maketitle

\begin{frame}[c]\frametitle{Outline}
    % \begin{multicols}{2}
	    \tableofcontents[sectionstyle=show/show, subsectionstyle=show/show, subsubsectionstyle=show/show]
    % \end{multicols}
\end{frame}

\section{Motivation}

\begin{frame}
\frametitle{Lecture goals}

    \begin{itemize}
        \item In this course, you may need a way to reproduce Lab exercises on your personal computers, possibly simulating a distributed environment

        \vfill{}

        \item Furthermore, you must be able to submit your homeworks/projects being sure they will run on other computer too
        \begin{itemize}
            \item Avoiding the situation \emph{``No, I swear, it used to work on my PC''} :)

            \item[!] Making your results more durable and \alert{reproducible}
        \end{itemize}

        \vfill{}

        \item You will learn some fundamentals of:
        \begin{itemize}
            \item[$\checkmark$] Build \& automation systems (e.g. \href{https://gradle.org/}{Gradle})
            \item[$\rightarrow$] Container Engines (e.g. \href{https://www.docker.com/}{Docker})
        \end{itemize}

        \vfill{}

        \item By reading \cite{envConf}, you can easily:
        \begin{itemize}
            \item[$\checkmark$] install Docker on your personal PC
            \item[$\rightarrow$] register a Docker ID with your \texttt{@studio.unibo.it} email address
        \end{itemize}
    \end{itemize}

\end{frame}

\section{Containers}

\begin{frame}
\frametitle{Containers}

    \begin{block}{}
        Containers are \alert{lightweight} virtual machines usually \emph{wrapping} a \alert{single application} and along with its \alert{environment}---there including environment variables, configuration files, network facilities, runtimes and operating systems. They can be deployed and transferred as a whole
    \end{block}
    %
    \begin{itemize}
        \item Containers are aimed at being easily deployable on any machine running a container \alert{engine}
        \item Such a machine is called \alert{host}
        \item Containers are \alert{lightweight} because the share the host kernel and some of its resources
        \item Containers are instances of some \alert{image}
        \item Images are serialisable and versioned and they are usually shared by means of public \alert{repositories}
    \end{itemize}


\end{frame}

\subsection{Why containers}

\begin{frame}
\frametitle{Why containers}

    \begin{itemize}
        \item We will exploit containers to easily deploy our software

        \item So will you, for your projects, if possible

        \item You may also use containers to simulate a distributed application on a single machine

        \item Containers are the building blocks of \alert{stacks}, which in turn are managed by \alert{orchestrators}, to deploy (\alert{micro})\alert{services}.
        This is what happens, for instance, behind the scenes of a \alert{Cloud} provider

    \end{itemize}

    \vspace{.3cm}

    \begin{block}{}
        \alert{From now on}, you are \alert{encouraged} to use containers to submit your projects
    \end{block}

\end{frame}

\subsection{Docker}

\begin{frame}
\frametitle{Docker}

    Docker is our Container Engine of choice, since it is the leading technology in this field
    %
    \vfill
    %
    \begin{itemize}
        \item It assumes images are created out of an existing application by means of \alert{\texttt{Dockefile}s}

        \begin{itemize}
            \item Think about an image as a shut-down OS where a single application -- along with all its dependencies -- is installed
        \end{itemize}

        \vfill

        \item Containers can be instantiated (i.e. \alert{run}) out of some pre-existing image by invoking a program which is assumed to be installed on that image

        \begin{itemize}
            \item Think about the instantiation process as turning on the OS and invoking that application
        \end{itemize}

    \end{itemize}

\end{frame}

\subsubsection{Images}

\begin{frame}
\frametitle[Docker Images]{Docker Images\hint{Image $\neq$ Picture}}

\begin{itemize}
	\item Images are executable files which, if run, instantiate a container
	\item Usually, you do not manipulate images directly: your local Docker installation takes care of pulling/pushing them from/to an online \alert{repository} for you:
	%
	\begin{itemize}
		\item[\$] \texttt{docker image} \texttt{\alert{<cmd>}}
	\end{itemize}
	%
	where \texttt{<cmd>} may be one of the following:
	%
	\begin{description}
		\item[\texttt{build}] | builds an image from a \alert{Dockerfile}
		\item[\texttt{ls}] | lists all locally available images
		\item[\texttt{pull}] | pulls an image from a \alert{repository} (you must be logged)
		\item[\texttt{push}] | push an image to a \alert{repository} (you must be logged)
		\item[\texttt{rm}] | remove one or more images
		\item[\texttt{tag}] | creates a symbolic tag for an image
		\item[\texttt{--help}] | shows the list of available commands for images

	\end{description}

\end{itemize}

\end{frame}

\begin{frame}
\frametitle{Docker Images -- Example}
	\begin{enumerate}
		\item Check the list of images currently available on your local machine (you may have none if you are running Docker for the first time)
		\begin{itemize}
		    \item[\$] \texttt{docker image \alert{ls}}
		\end{itemize}

		\item Download (or \alert{pull}) the \href{https://hub.docker.com/_/alpine/}{\texttt{alpine}} image from the Internet
		\begin{itemize}
		    \item[\$] \texttt{docker [image] \alert{pull} \textit{alpine}}
		    \item[] \hint{\texttt{[optional\_term] within square brackets}}
		    \item Where is the image being downloaded from?
		    \begin{itemize}
		        \item The default registry from \url{https://hub.docker.com}, you are assumed to have an account on there
		        \item thus, you own a repository too \texttt{https://hub.docker.com/u/\alert{<your\_username>}}
		    \end{itemize}
		    \item What's \texttt{alpine}?
		    \begin{itemize}
		        \item Alpine Linux\footnote{\url{https://alpinelinux.org/about}} is one of the most lightweight Linux distribution ever
		    \end{itemize}
		\end{itemize}

		\item Re-check the currently available images
		\begin{itemize}
		    \item[\$] \texttt{docker image \alert{ls}}
	    \end{itemize}
	\end{enumerate}
\end{frame}

\subsubsection{Containers}

\begin{frame}
\frametitle{Docker Containers I}

    \begin{itemize}
        \item Docker containers can be instantiated by means of the following syntax:
        %
        \begin{itemize}
            \item[\$] \texttt{docker [container] \alert{run} [\alert{<opts>}] \alert{\textit{<image>}} [\alert{<cmd>} [\textit{\alert{<args>}}]]}
        \end{itemize}
        %
        where :
        \begin{description}
            \item[\texttt{\textit{<image>}}] is the image to be instantiated,
            \item[\texttt{<cmd>}] is the command to be executed on the container, once instantiated (optional)
            \item[\texttt{\textit{<args>}}] is a sequence of space-separated arguments for \texttt{<cmd>}
            \item[\texttt{<opts>}] is a possibly empty sequence of options for the container instantiation. Several are available.
        \end{description}
        %
    \end{itemize}
\end{frame}

\begin{frame}
\frametitle{Docker Containers II}
    \begin{itemize}
        \item We will only exploit the following options in this lesson:
        %
        \begin{description}
            \item[\texttt{-i}] | runs the container in \emph{\alert{i}nteractive} mode

            \item[\texttt{-t}] | runs the container in \emph{\alert{t}erminal} emulation mode

            \item[\texttt{-d}] | runs the container in \emph{\alert{d}aemon} mode \hint{daemons = services}

            \item[\texttt{-e \textit{<key>}=<value>}] | sets an \emph{\alert{e}nvironment} variable \texttt{\textit{<key>}} to \texttt{<value>} on the container, before \texttt{<cmd>} is invoked

            \item[\texttt{-p \textit{<host>}:<guest>}] | \emph{\alert{p}ublish} port \texttt{<guest>} on the container to port \texttt{\textit{<host>}} on the host

            \item[\texttt{-h \textit{<hostname>}}]  |  sets the \emph{\alert{h}ostname} of the container to \texttt{\textit{<hostname>}}

            \item[\texttt{-v \textit{<local path>:<mount path>}}]  |  mounts the host's directory \texttt{<local path>} on the container, as a the \emph{\alert{v}olume}, into \texttt{\textit{<mount path>}}

            \item[\texttt{--name \textit{<name>}}]  |  assigns a unique \emph{\alert{n}ame} to the container
        \end{description}

    \end{itemize}
\end{frame}

\begin{frame}
\frametitle{Docker Containers III}
    \begin{itemize}

        \item As soon as \texttt{<cmd>} terminates its execution, the container is teminated too and any side effect applied to its storage is reverted

        \item If your application (\texttt{<cmd>}) requires NO user interaction and it CAN just run in the background, you must run it in daemon mode, otherwise in interactive mode

        \item In interactive mode, the application's \texttt{stdin}, \texttt{stdout}, \texttt{stderr} are redirected to/from your console
        %
        \begin{itemize}
            \item this is necessary if the containerised applicationion need to consume the users' inputs
        \end{itemize}

        \item If you need to interact with some container's shell, you should run it in terminal emulation mode
    \end{itemize}

\end{frame}

\begin{frame}[allowframebreaks]
\frametitle{Docker Containers -- Example}

    \begin{enumerate}
        \item Run a shell on a novel \texttt{alpine} container, in interactive mode
        %
        \begin{itemize}
            \item[\$] \texttt{docker run -i -e MY\_MSG="Hello World!" alpine sh}
            \begin{itemize}
                \item No, Docker is not freezed :)
                \item You simply started a \alert{very minimal} shell
            \end{itemize}
        \end{itemize}
        %
        \item To convince your self you are within a tiny Linux VM, try running
        %
        \begin{itemize}
            \item[\$] \texttt{cd ; whoami ; pwd ; hostname ; echo \$MY\_MSG}
        \end{itemize}
        %
        \item Cool. What else can a raw Alpine Linux do? Is even Java installed?
        %
        \begin{itemize}
            \item[\$] \texttt{java -version ; javac -version}
        \end{itemize}
        %
        \item No, Java? So bad. Containers have access to the internet (if the guest does too), so you can install software on them
        %
        \begin{itemize}
            \item[\$] \texttt{apk update ; apk add openjdk8} \hint{\href{https://wiki.alpinelinux.org/wiki/Alpine_Linux_package_management}{APK doc here}}
            \item[\$] \texttt{java -version ; javac -version}
        \end{itemize}

        \framebreak

        \item Try now to gracefully terminate the shell
        %
        \begin{itemize}
            \item[\$] \texttt{exit}
            \begin{itemize}
                \item the process launched by \texttt{docker run} terminated, so the container will be terminated too
            \end{itemize}
        \end{itemize}

        \item Re-run the same shell in terminal emulation mode
        %
        \begin{itemize}
            \item[\$] \texttt{docker run \alert{-t} -i -e MY\_MSG="Hello World!" alpine sh}
            \begin{itemize}
                \item Can you see the effect of option \texttt{-t}?
            \end{itemize}
        \end{itemize}

        \item Re-check whether Java is installed or not
        \begin{itemize}
            \item what do you expect?
        \end{itemize}

        \framebreak

        \item \alert{Before terminating this second container}, open another shell \alert{on the host} and run
        %
        \begin{itemize}
            \item[\$] \texttt{docker \alert{ps}}
            \begin{itemize}
                \item this command shows all \alert{currently} running containers
                \item you should be able to spot a line representing your container's info---there including a funny auto-generated name
            \end{itemize}
        \end{itemize}

        \item Try now to check for non-running containers too
        %
        \begin{itemize}
            \item[\$] \texttt{docker ps \alert{-a}}
            \begin{itemize}
                \item can you spot the previous container, the one you installed Java in?
            \end{itemize}
        \end{itemize}

        \item Terminate all containers, gracefully and then delete them all by running
        %
        \begin{itemize}
            \item[\$] \texttt{docker container \alert{prune}}
            \begin{itemize}
                \item now your environment is clean
            \end{itemize}
        \end{itemize}

    \end{enumerate}

\end{frame}

\subsubsection{Dockerfiles}

\begin{frame}%[allowframebreaks]
\frametitle{Dockerfiles}

    Dockerfiles\footnote{Lang reference: \url{https://docs.docker.com/engine/reference/builder}} are scripts aimed at \alert{building} new images
    %
    \begin{enumerate}
        \item An image is built by firstly specifying a \emph{base} image to start \alert{FROM}
        %
        \begin{itemize}
            \item in the simplest cases this is just a raw image such as \texttt{alpine} or \texttt{ubuntu}
            \item but serveral base images are available for most common runtimes, e.g. \href{https://hub.docker.com/r/anapsix/alpine-java/}{\texttt{alpine-java}} or \href{https://hub.docker.com/_/python/}{\texttt{python}}
        \end{itemize}
        \item Some files may be optionally \alert{COPY}ed on the novel image, at specific locations
        \item You can then specify some commands to be \alert{RUN} on the base image to install your application
        \item Some \alert{ENV}ironment variable may be optionally set
        \item Some transport-level port may be optionally \alert{EXPOSE} to the host or to the other containers
        \item Some default \alert{CMD} may optionally be configured to be invoke by default upon container start

    \end{enumerate}

\end{frame}

\begin{frame}[allowframebreaks]
\frametitle{Dockerfiles -- Example}

    \begin{enumerate}
        \item Conder the following Java program, and sopposed it has been properly Gradlefied
        %
        \lstinputlisting[language=Java]{./src/App1.java}

        \framebreak

        \item Create a new file named \alert{\texttt{Dockerfile}} withing the project root directory (say \alert{\texttt{my-app}}), containing the following lines:
        %
        \lstinputlisting[language=docker]{./src/Dockerfile1}

        \framebreak

        \item Build the new image by executing the following command:
        %
        \begin{itemize}
            \item[\$] \texttt{docker \alert{build} -t <DockerID>/my-app \alert{.}}\hint{pay attention to the dot}
            \begin{itemize}
                \item where \texttt{DockerID} is your username on \href{https://hub.docker.com/}{Dockerhub}
                \item and \alert{\texttt{<DockerID>/my-app}} is a \alert{tag} for the newly created image
            \end{itemize}
        \end{itemize}

        \item Now have a look to the list of available images on your computer
        %
        \begin{itemize}
            \item[\$] \texttt{docker image ls}
        \end{itemize}

        \item Can you spot the new entry? You can instantiate it as a \alert{named} container by typing
        %
        \begin{itemize}
            \item[\$] \texttt{docker run -t -i \alert{--name my-app} <DockerID>/my-app}
        \end{itemize}

        \item Run the app until it terminates. Can you understand how inputs can be provided to containerised applications?

        \framebreak

        \item From now on, you can re-run the app by typing
        %
        \begin{itemize}
            \item[\$] \texttt{docker [container] \alert{start} -i -a my-app}
        \end{itemize}

        \item Assuming you own an account on \href{https://hub.docker.com/}{DockerHub}, and your local Docker is properly \emph{logged in}, you can now \alert{push} your newly created image by typing
        %
        \begin{itemize}
            \item[\$] \texttt{docker [image] \alert{push} <DockerID>/my-app}
        \end{itemize}

        \item Now everybody can \alert{pull} your image for using it. You can try with your colleagues ones by typing
        %
        \begin{itemize}
            \item[\$] \texttt{docker [image] \alert{pull} <Colleague-DockerID>/my-app}
            \begin{itemize}
                \item What a great way to distribute software! :)
            \end{itemize}
        \end{itemize}

    \end{enumerate}

\end{frame}

%\section{Suggested Exercises}
%
%\begin{frame}%[allowframebreaks]
%\frametitle{Suggested Exercises}
%
%\begin{block}{Dockerify JEcho}
%    Try to Dockerify the JEcho application from previous Labs
%\end{block}
%
%\vfill{}
%
%\begin{block}{Dockerify \textsc{Linda} Service}
%    Try to Dockerify the \textsc{Linda} Service from previous Labs
%\end{block}
%
%\end{frame}

\section{About Docker Swarm}

\begin{frame}%[allowframebreaks]
    \frametitle{Docker Swarm Functionalities}


    \begin{itemize}
        \item Docker engines may be run in \emph{Swarm} mode
        %
        \begin{itemize}
            \item[\$] \texttt{docker swarm init{\color{gray}|join}}
        \end{itemize}

        \vfill

        \item Swarms are \alert{clusters} of machines running \emph{federated} Docker engines
        %
        \begin{itemize}
            \item nodes may be added to the cluster dynamically
            \item container may be executed/replicated on several nodes
        \end{itemize}

        \vfill

        \item In Swarm mode, Docker engines support more functionalities:
        %
        \begin{description}
            \item[services] | possibly replicated servers with automatic load distribution
            \item[networks] | virtual networks shared by serveral containers or services
            \item[stacks] | ensembles of services which must be deployed together, possibly in an orderly fashion
        \end{description}

        \vfill

        \item You may consider exploring these aspects if you want to
        %
        \begin{itemize}
            \item automate testing of your distributed system
            \item make your DS demo reproducible
        \end{itemize}

    \end{itemize}

\end{frame}

% \subsection{A Java Echo application}

% \begin{frame}[allowframebreaks]
% \frametitle{Exercise 2-1: Java Echo application}

%     \begin{enumerate}
%         \item Clone the Lab-2 GitLab repository at \url{https://gitlab.com/das-lab/courses/ds/aa1819/lab-2}
%         %
%         \begin{itemize}
%             \item it stores the Java Echo application within the \texttt{jecho} directory
%             \item which is a simple project: bare Java sources
%             \item the \texttt{im} directory is for the next exercise
%         \end{itemize}

%         \vspace{.5cm}

%         \item Inspect the Java Echo application source code and try to figure out its functioning
%         %
%         \begin{itemize}
%             \item what's its purpose from the user perspective?
%             \item does it requires some arguments? does they affect the application behaviour?
%             \item which assumptions (about the environment) does it relies upon?
%             \item[!] notice that the application requires network communication to occur on port 8080
%         \end{itemize}

%         \framebreak

%         \item Gradlefy the cloned application, making it compilable and runnable by means of \texttt{gradlew}
%         %
%         \begin{itemize}
%             \item you can \alert{generate} a Gradle Wrapper script by means of the following command:
%             %
%             \begin{itemize}
%                 \item[\$] \texttt{gradle wrapper}\hint{\href{https://docs.gradle.org/current/userguide/gradle_wrapper.html}{see Reference here}}
%             \end{itemize}


%             \item other files must be created manually :)

%             \item the canonical directory structure must be created manually :))

%             \item use \alert{environment variables} or Gradle properties to provide arguments to the application

%         \end{itemize}

%         \vspace{.5cm}

%         \item Dockerify the cloned application, creating an image out of it
%         %
%         \begin{itemize}
%             \item notice the application can run in two modes: either \emph{daemon} or \emph{gate}
%             %
%             \begin{itemize}
%                 \item you may need to actually create two images, one for each mode
%             \end{itemize}

%             \item remember to \alert{expose} the 8080 port! \hint{\href{https://docs.docker.com/engine/reference/builder/\#expose}{Reference here}}

%             \item remember to tag the images with the \texttt{\textit{<DockerID>}/jecho-daemon} and \texttt{\textit{<DockerID>}/jecho-gate} tags
%         \end{itemize}

%         \framebreak

%         \item Run the newly created images as containers, named \texttt{jecho-d} and \texttt{jecho-g} respectively
%         %
%         \begin{itemize}
%             \item in case of \texttt{jecho-d}, remember to \alert{publish} the container's 8080 port on the host's 8888 port \hint{\href{https://docs.docker.com/engine/reference/commandline/run/\#options}{Reference here}}

%             \item in case of \texttt{jecho-g}, it need to be provided with the IP of \texttt{jecho-d}

%             \item you can \alert{inspect} a container in order to find its IP:
%             %
%             \begin{itemize}
%                 \item[\$] \texttt{docker container \alert{inspect} jecho-d} \hint{\href{https://docs.docker.com/engine/reference/commandline/container_inspect/}{Reference here}}
%             \end{itemize}

%         \end{itemize}

%         \vspace{.5cm}

%         \item Try interacting with \texttt{jecho-d}, both from the host and from \texttt{jecho-g}

%         \vspace{.5cm}

%         \item Describe your tests into the \texttt{README.md} file
%     \end{enumerate}


% \end{frame}

% \subsection{IM application}

% \begin{frame}%[allowframebreaks]
% \frametitle{Exercise 2-2: IM application, again}

%     \begin{enumerate}
%         \item Gradlefy the IM application from Lab-1
%         %
%         \begin{itemize}
%             \item use environment variables to provide arguments
%             \item store the source code within the \texttt{im} directory
%             \item you can use the provided solution to the Lab-1 exercise, if you didn't completed it
%         \end{itemize}

%         \item Dockerify it

%         \item Instantiate at least 2 containers, each one wrapping an instance of the IM application
%         %
%         \begin{itemize}
%             \item inter-connect them by means of containers' IPs
%         \end{itemize}

%         \item Use 2 disjoint consoles to emulate a multi-IM scenario

%         \item Describe your tests into the \texttt{README.md} file

%         \vspace{1cm}

%         \item[!] Congrats! You are emulating a distributed system on your local PC! :)
%     \end{enumerate}

% \end{frame}

%===============================================================================
\section*{}
%===============================================================================
\frame{\titlepage}

%===============================================================================
% \section*{\bibname}
%===============================================================================

\setbeamertemplate{page number in head/foot}{}
%\\\\\\\\\\\\\\\\\\\\\
%\begin{frame}[t,allowframebreaks,noframenumbering]{\refname}
\begin{frame}[c]{\refname}
    %	\footnotesize
    	\scriptsize
    %\tiny
    \bibliographystyle{plain}
    \bibliography{sd-lab-containers}
\end{frame}
%\\\\\\\\\\\\\\\\\\\\\

%%%%%%%%%%%%%%%%%%%%%%%%%%%%%%%%%%%%%%%%%%%%%%%%%%%%%%%%%%%%%%%%%%%%%%%%%%%%%%%
\end{document}
%%%%%%%%%%%%%%%%%%%%%%%%%%%%%%%%%%%%%%%%%%%%%%%%%%%%%%%%%%%%%%%%%%%%%%%%%%%%%%%%
